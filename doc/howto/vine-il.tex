\section{The Vine Intermediate Language}
\label{sec:vine-language}

% Non-terminals from a grammar, suitable for use in math mode.
\newcommand{\nt}[1]{\text{\em #1}}

% Terminals from a grammar, suitable for use in math mode.
\newcommand{\term}[1]{\text{\tt #1}}

\begin{table}
\centering
\begin{footnotesize}
\begin{tabular}{lll}
  \nt{program}&::=&
        \nt{decl}* \nt{stmt}*\\

  \nt{decl}&::=& 
         \term{var} \nt{var}\term{;}\\

  \nt{stmt}&::=&  
         \nt{lval} \term{=} \nt{exp}\term{;}
     $|$ \term{jmp}\term{(}\nt{exp}\term{)}\term{;}
     $|$ \term{cjmp}\term{(}\nt{exp}\term{,} \nt{exp}\term{,} \nt{exp}\term{);}
     $|$ \term{halt}\term{(}\nt{exp}\term{)}\term{;}
     $|$ \term{assert}\term{(}\nt{exp}\term{)}\term{;}\\
     & &  $|$ \term{label} \nt{label}\term{:}
     $|$ \term{special} \nt{string}\term{;}
     $|$ \term{\char"7B} \nt{decl}* \nt{stmt}*\term{\char"7D}\\

  \nt{label}&::=&
         \nt{identifier}\\

  \nt{lval}&::=&
         \nt{var}
     $|$ \nt{var}\term{[}\nt{exp}\term{]}\\

  \nt{exp}&::=&
         \term{(} \nt{exp} \term{)}
     $|$ \nt{lval}
     $|$ \term{name}\term{(}\nt{label}\term{)}
     $|$ \nt{exp} $\Diamond_b$ \nt{exp} 
     $|$ $\Diamond_u$ \nt{exp}
     $|$ \nt{const}\\
     & &  $|$ \term{let} \nt{lval} \term{=} \nt{exp} \term{in} \nt{exp}
     $|$ \term{cast}\term{(}\nt{exp}\term{)}\nt{cast{\tt \_}kind}\term{:}$\tau_{\text{reg}}$\\

  \nt{cast{\tt \_}kind}&::=&  
         \term{Unsigned} $|$ \term{U}
     $|$ \term{Signed} $|$ \term{S}
     $|$ \term{High} $|$ \term{H}
     $|$ \term{Low} $|$ \term{L}\\

  \nt{var}&::=& 
         \nt{identifier}\term{:}$\tau$\\


  $\Diamond_b$&::=&
         \term{+} $|$ \term{-} $|$ \term{*}
     $|$ \term{/} $|$ \term{/\$} $|$ \term{\%} $|$ \term{\%\$}
     $|$ \term{<<} $|$ \term{>>} $|$ \term{\char"40 >>}
     $|$ \term{\&} $|$ \term{\char"5E} $|$ \term{|} \\
     & & $|$ \term{==} $|$ \term{<>}
     $|$ \term{<} $|$ \term{<=} $|$ \term{>} $|$ \term{>=}
     $|$ \term{<\$} $|$ \term{<=\$} $|$ \term{>\$} $|$ \term{>=\$}\\

  $\Diamond_u$&::=& \term{-} $|$ \term{!}\\

  \nt{const}&::=& \nt{integer}:$\tau_{\text{reg}}$ \\

  \nt{$\tau$} &::=&
          \nt{$\tau_{\text{reg}}$} 
      $|$ \nt{$\tau_{\text{mem}}$}\\
  
  \nt{$\tau_{\text{reg}}$}&::=&
          \term{reg1\_t} 
      $|$ \term{reg8\_t} 
      $|$ \term{reg16\_t} 
      $|$ \term{reg32\_t} 
      $|$ \term{reg64\_t}\\

  \nt{$\tau_{\text{mem}}$}&::=& 
          \term{mem32l\_t}
      $|$ \term{mem64l\_t}
      $|$ \nt{$\tau_{\text{reg}}$}\term{[}\nt{const}\term{]}\\
\end{tabular}
\end{footnotesize}
\caption{The grammar of the Vine Intermediate Language (IL).}
\label{vine:syntax}
\end{table}

The Vine IL
is the target language during lifting, as well as the analysis
language for back-end program analysis.  The semantics of the IL are designed
to be faithful to assembly languages.  Table~\ref{vine:syntax} shows
the syntax of Vine IL. 
The lexical syntax of identifiers and strings are as in C.
Integers may be specified in decimal, or in hexadecimal with a prefix
of {\tt 0x}.
Comments may be introduced with {\tt //}, terminated by the end of a
line, or with {\tt /*}, terminated by {\tt */}.

The base types in the Vine IL are 1, 8, 16, 32, and 64-bit-wide bit
vectors, also called registers.
1-bit registers are used as booleans; {\tt false} and {\tt true} are
allowed as syntactic sugar for {\tt 0:reg1\_t} and {\tt 1:reg1\_t}
respectively.
There are also two kinds of aggregate types, which we call {\em arrays}
and {\em memories}.
Both are usually used to represent the memory of a machine, but at
different abstraction levels.
An array consists of distinct elements of a fixed register type,
accessed at consecutive indices ranging from 0 up to one less than
their declared size.
By contrast, memory indices are always byte offsets, but memories may
be read or written with any type between 8 and 64 bits.
Accesses larger than a byte use a sequence of consecutive bytes, so
accesses at nearby addresses might partially overlap, and it is
observable whether the memory is little-endian (storing the least
significant byte at the lowest address) or big-endian (storing
the most significant byte at the lowest address).
Generally, memories more concisely represent the semantics of
instructions, but arrays are easier to analyze, so Vine analyses will
convert memories into arrays, a process we sometimes call {\em
de-endianization}.
The current version of Vine supports two little-endian memory types,
with either 32-bit or 64-bit address sizes.

Expressions in Vine are side-effect free.
Variables and constants must be labeled with their type (separated
with a colon) whenever they appear.
The binary and unary operators are similar to those of C, with the
following differences:
\begin{itemize}
\item Not-equal-to is {\tt <>}, rather than {\tt !=}.
\item The division, modulus, right shift, and ordered comparison
  operators are explicitly marked for signedness: the unadorned
  versions are always unsigned, while the signed variants are suffixed
  with a {\tt \$} (for ``signed''), or in the case of right shift
  prefixed with an {\tt\char"40} (for ``arithmetic'').
\item There is no distinction between logical and bitwise
  operators, so {\tt \&} also serves for {\tt \&\&}, {\tt |} also
  serves for {\tt ||}, and {\tt !} also serves for {\tt \char"7E}.
\end{itemize}
There is no implicit conversion between types of different widths;
instead, all conversions are through an explicit cast operator that
specifies the target type.
Widening casts are either {\tt Unsigned} (zero-extending) or {\tt
Signed} (sign-extending), while narrowing casts can select either the
{\tt High} or {\tt Low} portion of the larger value.
(For brevity, these are usually abbreviated by their first letters.)
A {\tt let} expression, as in functional languages, allows the
introduction of a temporary variable.

A program in Vine is a sequence of variable declarations, followed by
a sequence of statements; block structure is supported with curly
braces.
(In fact, the parser allows declarations to be intermixed with
statements, but the effect is as if the declarations had all appeared
first.)
Some documents also refer to statements as ``instructions,'' but note
that more complex machine instructions translate into several Vine
statements.
The most frequent kind of statement is an assignment to a variable or
to a location in an array or memory variable.
Control flow is unstructured, as in assembly language: program
locations are specified with labels, and there are unconditional ({\tt
jmp}) and conditional ({\tt cjmp}) jumps.
The argument to {\tt jmp} and the second and third arguments to {\tt
cjmp} may be either labels (introduced by {\tt name}), or a register
expression to represent a computed jump.
The first argument to {\tt cjmp} is a {\tt reg1\_t} that selects the
second (for 1) or third (for 0) argument as the target.

A program can halt normally
at any time by issuing the {\tt halt} statement.  We also provide {\tt
  assert}, which acts similar to a C assert: the asserted expression
must be true, else the machine halts.
A {\tt special} in Vine corresponds to a call to an externally
defined procedure or function.  The argument of a special indexes what kind
of special, e.g., what system call.  The semantics of {\tt special} is
up to the analysis; its operational semantics are not defined.  We
include {\tt special} as an instruction type to explicitly distinguish
when such calls may occur that alter the soundness of an analysis. A
typical approach to dealing with {\tt special} is to replace {\tt
special} with an analysis-specific summary written in the
Vine IL that is appropriate for the analysis.

