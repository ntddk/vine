\subsection {Generating the IL and the STP formula}

We start with a trace (generated, for instance, by TEMU) that records
the instructions executed on a program run, the data values they
operated on, and which data values were derived from a distinguished
set of (``tainted'') input values. We're going to do operations where
we consider that input to be a symbolic variable, but the first step
is to interpret the trace. The x86 instructions in the trace are a
pretty obscure representation of what is actually happening in the
program, so we'll translate them into a cleaner intermediate language
(IL, abbreviated IR in command options).

First, let us check if we have got a meaningful trace.  One way to do
so is to print the trace, and see that at least the expected
instructions are marked as tainted.  For this, you may use the
\texttt{trace\_reader} command utility in Vine. As shown below, in the
output you should be able to see the compare instruction
that comapares the input to the immediate value 5. The presence of
tainted operands in any instruction are indicated by the record
containing ``T1''.

\begin{Verbatim}[frame=lines, framesep=.5em]

% cd bitblaze/vine
% ./trace_utils/trace_reader -trace examples/five.trace | grep T1
...
...
804845a:       cmpl    $0x5,-0x4(%ebp)   I@0x00000000[0x00000005] \
       T0      M@0xbffffac4[0x00000005]        T1 {15 (1001, 0) (1001, 0) \
	 (1001, 0) (1001, 0) } 

\end{Verbatim}
%$

Of course, the real output of that command contains many of
instructions, but we've picked out a key one: an instruction from the
main program (you can tell because the address is in the \verb'0x08000000'
range) in which a value from the stack (\verb'-0x4(%ebp)') is
compared (a \verb'cmpl' instruction) with a constant integer 5
(\verb'$0x5').%$
The later fields on the line represent the instruction operands and
their tainting.

We can then use the \texttt{appreplay} utility to
both convert the trace into IL for and then to generate an STP formula
given the constraints on the symbolic input. The invocation looks
like:
\begin{Verbatim}[frame=lines, framesep=.5em]
% ./trace_utils/appreplay -trace examples/five.trace \
  -stp-out five.stp -ir-out five.il -wp-out five.wp
...
Time to create sym constraint from TM: 0.288464
\end{Verbatim}

This command line produces the final STP file as \verb'foo.stp', and
the intermediate files \verb'foo.il' and \verb'foo.wp' to demonstrate
the steps of the processing.
Remember that Vine
uses its own IL to model the semantics of instructions in a
simpler RISC-like form.  The IL and WP output files are in this IL
language. If you aren't interested in these files, you can omit the
\verb'-ir-out' and \verb'-wp-out' options.
You can learn about other options that may be supplied to {\tt
appreplay} in Section~\ref {sec:utils}.

In essence, \texttt{appreplay} models the logic of the executed
instructions, generating a path constraint needed to force the
execution down the path taken in the trace.  A variable \verb'post' is
introduced, which is the conjunction of the conditions seen along the
path. In the file \verb'foo.il', you can see this variable is assigned
at each conditional branch point as $post = post \wedge condition$,
where a condition is a variable modeling the compare operation's
result that must be true to force execution to continue along the path
taken. (Because the language is explicitly typed and \verb'appreplay'
is careful to generate unique names, the full name of the \verb'post'
variable is likely something like \verb'post_1034:reg1_t', where the
part after the colon tells you it's a one-bit (boolean) variable.)

This weakest precondition formula is then converted to the format of
the STP solver's input.
